\section{Maximum Matchings}
\label{sec:max_matchings}

\newcommand{\maxmatch}{\hyperref[prob:max_matching]{\color{black}\textsf{Maximum Matching}}}
\newcommand{\symdiff}{\ensuremath{\bigtriangleup}}

%\subsection{Introduction and Notation}

\begin{definition}[Matching]
    Given a graph $G = (V, E)$, a matching $M \sse E$ is a set of edges such that no two edges in $M$ share 
    a common vertex. 
    \label{def:matching}
    \begin{definition}[Covered/Exposed]
        A node is $M$-covered if some edge in $M$ is incident to it. Else it is $M$-exposed. 
    \end{definition}
\end{definition}

Note that a matching $M$ covers exactly $2\abs{M}$ nodes, leaving $\abs{V} - 2\abs{M}$ nodes exposed. 
A matching is \emph{perfect} if it covers all the nodes. A basic decision problem is to decide if a graph has a perfect matching. 
A more general problem is to find a maximum matching. 

\begin{problem}[Maximum (Cardinality) Matching]
    Given a graph $G = (V, E)$, find a matching $M$ that has maximum cardinality. Equivalently, find a 
    matching with the fewest exposed nodes. 
    \label{prob:max_matching}
\end{problem}

\subsection{Augmenting Paths}

A path $P$ is a collection of edges $\lrp{v_0, v_1}, \ldots, \lrp{v_{k-1}, v_k} = e_1, \ldots, e_k $ where each $v_i$ is distinct.   

\begin{definition}[Alternating Path]
    Given a matching $M$ in a graph $G$, a path $P$ in $G$ is $M$-alternating if it alternates
    between edges in $M$ and edges in $E \sm M$. 
    \label{def:alternate_path}
\end{definition}

\begin{definition}[Augmenting Path]
    Given a matching $M$ in a graph $G$, a path $P$ in $G$ is $M$-augmenting if it is $M$-alternating and  
    its end nodes are distinct and $M$-exposed. 
    \label{def:augment_path}
\end{definition}

Augmenting paths can be used to identify a \maxmatch{} in the following way. 

\begin{theorem}[Augmenting Path Theorem]
    A matching $M$ in a graph $G$ is maximum iff there is no $M$-augmenting path. 
    \label{thm:augmenting_path}
\end{theorem}
\begin{proof}
    $\Longrightarrow$ (by contrapositive): Suppose there exists some $M$-augmenting path $P$ with end nodes $u, v$. 
    Then $M^\prime = M \bigtriangleup P$ (i.e.\! the matching constructed by adding the unmatched edges and removing the matched edges of $P$)
    covers all the nodes covered by $M$ plus nodes $u, v$. Thus, $M$ is not a maximum matching. 
\end{proof}
\begin{proof}
    $\Longleftarrow$ (by contrapositive): Let $M^{\ast}$ be a maximum matching. Let $Q = M^{\ast} \symdiff M$. 
    Then each node is incident to at most one edge in $M^{\ast} \cap Q$ and one edge in $M \cap Q$. Hence, $Q$ is an edge set 
    of node disjoint paths and circuits where edges alternate between belonging in $M^{\ast}$ and $M$. 
    Because the edges are taken from matchings, all circuits must be of even length 
    and contain the same number of edges from $M^{\ast}$ and $M$. Therefore, since $\abs{M^{\ast}} > \abs{M}$, 
    there must be at least one path in $Q$ that contains more edges from $M^{\ast}$ than $M$. Such a path is 
    $M$-augmenting. 
\end{proof}

\subsection{Alternating Trees}

Let $M$ be a current matching and $X$ the set of $M$-exposed nodes. 

\begin{definition}[Alternating Tree]
    An $M$-alternating tree is a tree $T$ with root node $r \in X$ such that along every path to a leaf node $v$, 
    the path is $M$-alternating with $e_i \in M$ iff $i = 2k$ for some $k \in \Z^+$. 
    \label{def:alternating_tree}
\end{definition}
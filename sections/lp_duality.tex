\section{LP-Duality}
\label{sec:lp_duality}

\newcommand{\primal}{\hyperref[def:primal_dual]{$(P)$}}
\newcommand{\dual}{\hyperref[def:primal_dual]{$(D)$}}

For a linear program in canonical form, the goal is to find a non-negative, rational vector $x \in \Q^n$ that minimizes
a given linear objective function in $x$ subject to some linear constraints on $x$.
The coefficients of the linear objective function can be represented by a vector $c \in \Q^n$, 
the coefficients of the linear constraints can be represented by a matrix $A = (a_{ij}) \in \Q^{m \times n}$, 
and the values of the linear constraints can be represented by a vector $b \in \Q^m$. 

\begin{definition}[Primal and Dual]
    Given a linear programming problem in canonical form denoted as $(P)$, we can induce a problem denoted 
    by $(D)$ with the following form:\\
    \label{def:primal_dual} 
    \begin{minipage}{0.5\linewidth}
        \begin{mini*}
            {}{\sum_{j=1}^{n} c_j x_j}{}{(P)\quad}{}
            \addConstraint{\sum_{j=1}^{n} a_{ij} x_j}{\geq b_i}{\quad i=1, \ldots, m}
            \addConstraint{x_j}{\geq 0}{\quad j = 1, \ldots, n}
        \end{mini*}
        \;
    \end{minipage}%
    \begin{minipage}{0.5\linewidth}
        \begin{maxi*}
            {}{\sum_{i=1}^{m} b_i y_i}{}{(D)\quad}{}
            \addConstraint{\sum_{j=1}^{m} a_{ij} y_i}{\leq c_j}{\quad j=1, \ldots, n}
            \addConstraint{y_i}{ \geq 0}{\quad i = 1, \ldots, m}
        \end{maxi*}
        \;
    \end{minipage}  
    where $a_{ij}, b_i,$ and $c_i$ are given rational numbers and $y_i$ corresponds to the $i$th inequality of $(P)$.
    \begin{definition}[Primal]
        The problem \primal{} is referred to as the \emph{primal}. 
        \label{def:primal}
    \end{definition}
    \begin{definition}[Dual]
        The \emph{dual} of the primal is problem \dual{}. 
        \label{def:dual}
    \end{definition}
\end{definition}

Note that the dual of a dual program is the primal program. 
Every feasible solution for the dual serves as a lower bound on the optimal objective function 
value of the primal. The reverse also holds in that every feasible solution to the primal 
serves as an upper bound on the optimal objective function value of the dual. 

\begin{theorem}[Weak Duality]
    If $\bm{x} = \lrp{x_1, \ldots, x_n} $ is a feasible solution to the LP \primal{} and $\bm{y} = \lrp{y_1, \ldots, y_m} $ a feasible solution to the LP \dual{}, 
    then $\sum_{j=1}^{n} c_j x_j \geq \sum_{i=1}^{m} b_i y_i$. 
    \label{thm:weak_duality}
\end{theorem}
\begin{proof}
    \begin{align*}
        \sum_{j=1}^{n} c_j x_j \geq \sum_{j=1}^{n} \lrp{\sum_{i=1}^{m} a_{ij} y_i} x_j
        =  \sum_{i=1}^{m} \lrp{\sum_{j=1}^{n} a_{ij} x_j} y_i 
        \geq \sum_{i=1}^{m} b_i y_i
    \end{align*}
\end{proof}

As a consequence of \cref{thm:weak_duality}, if we find that there exist some 
feasible solutions to \primal{} and \dual{} that have matching objection function values, 
then these solutions must be optimal. 

\begin{theorem}[Strong Duality/LP-Duality]
    If the LPs \primal{} and \dual{} are both feasible, and $\bm{x^{\ast}} = \lrp{x_1^{\ast}, \ldots, x_n^{\ast}}$
    and $\bm{y^{\ast}} = \lrp{y_1^{\ast}, \ldots, y_m^{\ast}}$ are optimal solutions to \primal{} and \dual{}, respectively, 
    then $\sum_{j=1}^{n} c_j x_j = \sum_{i=1}^{m} b_i y_i$. 
    \label{thm:strong_duality}
\end{theorem}

